\documentclass[10pt,twocolumn,letterpaper]{article}

\usepackage{cvpr}
\usepackage{times}
\usepackage{epsfig}
\usepackage{graphicx}
\usepackage{amsmath}
\usepackage{amssymb}

% Include other packages here, before hyperref.

% If you comment hyperref and then uncomment it, you should delete
% egpaper.aux before re-running latex.  (Or just hit 'q' on the first latex
% run, let it finish, and you should be clear).
%\usepackage[pagebackref=true,breaklinks=true,letterpaper=true,colorlinks,bookmarks=false]{hyperref}

\cvprfinalcopy % *** Uncomment this line for the final submission

%\def\cvprPaperID{****} % *** Enter the CVPR Paper ID here
%\def\httilde{\mbox{\tt\raisebox{-.5ex}{\symbol{126}}}}

% Pages are numbered in submission mode, and unnumbered in camera-ready
\ifcvprfinal\pagestyle{empty}\fi
\begin{document}

%%%%%%%%% TITLE
\title{Breadboard to Schematic}

\author{Catherine Olsson \\
MIT\\
{\tt\small catherio@mit.edu}
% For a paper whose authors are all at the same institution,
% omit the following lines up until the closing ``}''.
% Additional authors and addresses can be added with ``\and'',
% just like the second author.
% To save space, use either the email address or home page, not both
\and
Michele Pratusevich\\
MIT\\
{\tt\small mprat@mit.edu}
}

\maketitle
\thispagestyle{empty}

%%%%%%%%% ABSTRACT
\begin{abstract}

TODO: Write the abstract.

\end{abstract}

%%%%%%%%% BODY TEXT
\section{Introduction}

Breadboards are an important tool for do-it-yourself hardware designers to
quickly test circuit systems. They are easy to assemble, easy to change, and
easy to test, so are the tool of choice for the first prototype of many
hobbyists and students. A programmatically-drawn, or more likely hand-drawn,
schematic diagram is used to prototype the circuit board, but if the circuit is
going to be used for high-speed, low-noise, or multiple-production
applications, a printed circuit board (PCB) is much more desirable than a
breadboard. Our approach to solve this problem was a vision-based tool designed
to go from a picture of a breadboard circuit to a file written in a PCB-ready
format given by Eagle. Through a variety of approaches, we found the task too
large and have implemented pieces of the full pipeline. Them main goal of the
project was to transform the information about the circuit that we had in image
form (pixel representation) to a different more general representation
(component representation). 

%-------------------------------------------------------------------------

\section{Related Work}

TODO

%-------------------------------------------------------------------------
\section{The Approach}

The problem of translating information from a picture of a circuit to a
PCB-friendly file format has two main steps. First, the components must be
identified in the image, and second, the components must be placed in a virtual
grid representation independent of their relation in the image. 

For our implementation we chose to use python and it's Tkinter, PIL, SciPy, and
Numpy packages for ease of implementation, user interface, and speed. 

%-------------------------------------------------------------------------
\subsection{Segmentation into Circuit Components}

\subsubsection{Color Following}

As a first step to tackling the component segmentation problem, we took
advantage of the natural segmentation in a breadboard picture according to
color. Components on a breadboard are different colors that are localized in
one location, so to take advantage of this, we developed an algorithm that
``grows'' a component based on a given color pixel. A seed color is given to
the algorithm 

TODO 

\subsubsection{k-Means Clustering}

TODO

\subsection{Virtual Grid Representation}

TODO

%-------------------------------------------------------------------------
\section{Results}

TODO

%-------------------------------------------------------------------------
\section{Conclusion}

TODO

{\small
\bibliographystyle{ieee}
\bibliography{egbib}
}

\end{document}
